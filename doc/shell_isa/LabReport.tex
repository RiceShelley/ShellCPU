\documentclass{article}
\usepackage{lib/setup}
\addbibresource{mybibliography.bib}


%%%%%%%%%%%%%%%%%%%%%%%%%%%%%%%%%%%%%%%%%%%%%%%%%%%%%%%%%%%%%%%%%%%%%%%%%%%%%%%%
%%%%%%%%%%%%%%%%%%%%%%%%%%%%%%%%%%%%%%%%%%%%%%%%%%%%%%%%%%%%%%%%%%%%%%%%%%%%%%%%
%%%%%%%%%%%%%%%%%%%%%%%%%%%%%%%%%%%%%%%%%%%%%%%%%%%%%%%%%%%%%%%%%%%%%%%%%%%%%%%%
% Set up your information
%%%%%%%%%%%%%%%%%%%%%%%%%%%%%%%%%%%%%%%%%%%%%%%%%%%%%%%%%%%%%%%%%%%%%%%%%%%%%%%%
\subtitle{Instruction Set Architecture}
\title{The Shell}
\author{Jonathan Rice Shelley II}
\date{\today}
\headerTitle{Shell ISA}
\subject{Version 1.0}

%%%%%%%%%%%%%%%%%%%%%%%%%%%%%%%%%%%%%%%%%%%%%%%%%%%%%%%%%%%%%%%%%%%%%%%%%%%%%%%%
%%%%%%%%%%%%%%%%%%%%%%%%%%%%%%%%%%%%%%%%%%%%%%%%%%%%%%%%%%%%%%%%%%%%%%%%%%%%%%%%
%%%%%%%%%%%%%%%%%%%%%%%%%%%%%%%%%%%%%%%%%%%%%%%%%%%%%%%%%%%%%%%%%%%%%%%%%%%%%%%%


% BEGIN DOCUMENT
\begin{document}
\maketitle

\section{Contents}
\begin{par}
	\begin{enumerate}
	\item \nameref{overview}
	\item \nameref{regmap}
	\item \nameref{uins}
	\item \nameref{id}
	\item \nameref{progref} 
	\item \nameref{commonC}
	\item \nameref{insuj}
	\end{enumerate}
\end{par}

\section{Shell ISA Overview}
\label{overview}
\begin{par}
	The Shell ISA follows RISC design principles and is intended for a new 16-bit microprocessor. The primary goals are low cost implementation and minimal clock cycles per instruction. The Shell ISA contains 8 unique instructions encoded by a 3-bit op code. Multiple encodings of instructions results in 23 assembly instructions made available to the programmer. 16 bit data words (2's complement) and instructions are used exclusively in the microprocessor design. The programmer has access to 7 programmable registers in addition to an 8th register that is always zero. All programmer accessible registers are one word in length or 16 bits. Linear addressing of 1K 16-bit word-addressable only memory is supported. 
\end{par}

\section{Register mapping and suggested usage conventions}
\label{regmap}
\begin{par}
	The table below list the user addressable registers and their intended use. In addition to the 8 registers provided, the programmer can also interact with the stack pointer register via the lsp (load stack pointer) instruction. The programmer is only required to initialize the stack pointer updating the SP's value on push and pull is handled in hardware. 
	\begin{center}
		\begin{tabular}{|c|c|c|}
			\hline
			\textbf{GPR address} & \textbf{Register} & \textbf{Description} \\
			\hline 
			000 & r0 & always zero \\
			\hline 
			001 & r1 & general purpose \\
			\hline 
			010 & r2 & general purpose \\
			\hline 
			011 & r3 & general purpose \\
			\hline 
			100 & r4 & general purpose \\
			\hline 
			101 & r5 & size of stack frame \\
			\hline 
			110 & r6 & subroutine return register \\
			\hline 
			111 & r7 & subroutine return address \\
			\hline 
		\end{tabular}
	\end{center}		
\end{par}

\newpage

\section{Unique instructions}
\label{uins}
\begin{par}
	The 8 unique instructions supported by the ISA are listed below. 
	\begin{center}
		\begin{tabular}{|c|c|}
			\hline
			\textbf{Instruction opcode} & \textbf{Description} \\
			\hline 
			000 & System commands (halt) \\
			\hline 
			001 & Branching instructions \\
			\hline
			010 & Jump and link register \\
			\hline
			011 & Load upper immediate \\
			\hline
			100 & Store word \\
			\hline
			101 & Load word \\
			\hline
			110 & ALU instructions \\
			\hline
			111 & Stack instructions \\
			\hline
		\end{tabular}
	\end{center}
	
\end{par}

\section{Instruction definitions}
\label{id}
\begin{par}
	All hardware instructions are listed below by their instruction type. x's in bit fields denote a value assigned by the programmer. 
	\subsection{RRR-Type instructions}
	The RRR-type instruction group is responsible for encoding instructions that require 3 registers. The instruction type has a 4 bit function field to support additional formats. The majority of RRR-type instructions are ALU related. 
	\begin{center}
		\begin{tabular}{|c|c|c|c|c|c|}
			\hline 
			& \multicolumn{5}{|c|}{\textbf{16 bit instruction encoding}} \\ 
			\hline
			\textbf{Mnemonic \& operands } & \textbf{OP code} & \textbf{Func 4} & \textbf{Dest register} & \textbf{Register operand 1} & \textbf{Register operand 2} \\
			\hline 
			add $ R_{1} $, $ R_{2} $, $ R_{3} $ & 110 & 0000 & xxx & xxx & xxx \\ 
			\hline 
			sub $ R_{1} $, $ R_{2} $, $ R_{3} $ & 110 & 0001 & xxx & xxx & xxx  \\ 
			\hline 
			and $ R_{1} $, $ R_{2} $, $ R_{3} $ & 110 & 0010 & xxx & xxx & xxx  \\ 
			\hline 
			or $ R_{1} $, $ R_{2} $, $ R_{3} $ & 110 & 0011 & xxx & xxx & xxx  \\ 
			\hline 
			xor $ R_{1} $, $ R_{2} $, $ R_{3} $ & 110 & 0100 & xxx & xxx & xxx  \\ 
			\hline 
			nand $ R_{1} $, $ R_{2} $, $ R_{3} $ & 110 & 0101 & xxx & xxx & xxx  \\ 
			\hline 
		\end{tabular} 
	\end{center}

	\newpage

	\subsection{RR-Type instructions}
	The RR-type instruction group encodes two registers. The instruction type has a 4 bit function field to support additional instruction encodings. 
	\begin{center}
		\begin{tabular}{|c|c|c|c|c|c|}
			\hline 
			& \multicolumn{5}{|c|}{\textbf{16 bit instruction encoding}} \\ 
			\hline
			\textbf{Mnemonic \& operands} & \textbf{OP code} & \textbf{Function 4} & \textbf{Destination register} & \textbf{Source register} & \textbf{Zero padding} \\
			\hline 
			lw $ R_{1} $, $ R_{2} $ & 101 & 0000 & xxx & xxx & 000 \\ 
			\hline 
			sw $ R_{1} $, $ R_{2} $ & 100 & 0000 & xxx & xxx & 000 \\ 
			\hline 
			asr $ R_{1} $, $ R_{2} $ & 110 & 0110 & xxx & xxx & 000 \\ 
			\hline
			asl $ R_{1} $, $ R_{2} $ & 110 & 0111 & xxx & xxx & 000 \\ 
			\hline
			cmp $ R_{1} $, $ R_{2} $ & 110 & 1010 & xxx & xxx & 000 \\ 
			\hline
			jalr $ R_{1} $, $ R_{2} $ & 010 & 0000 & xxx & xxx & 000 \\ 
			\hline
		\end{tabular} 
	\end{center}

	\subsection{R-Type instructions}
	The R-type instruction group encodes one register. The instruction group has a 4 bit function field to support additional instruction encodings. 
	\begin{center}
		\begin{tabular}{|c|c|c|c|c|c|}
			\hline 
			& \multicolumn{5}{|c|}{\textbf{16 bit instruction encoding}} \\ 
			\hline
			\textbf{Mnemonic \& operands} & \textbf{OP code} & \textbf{Function 4} & \textbf{Operand register} & \multicolumn{2}{|c|}{\textbf{Zero padding}} \\
			\hline 
			push $ R_{1} $ & 111 & 0000 & xxx & \multicolumn{2}{|c|}{000000} \\ 
			\hline 
			pop $ R_{1} $ & 111 & 0001 & xxx & \multicolumn{2}{|c|}{000000} \\ 
			\hline 
			lsp $ R_{1} $ & 111 & 0010 & xxx & \multicolumn{2}{|c|}{000000} \\ 
			\hline 
		\end{tabular} 
	\end{center}

	\subsection{RI-Type instructions}
	The RI-type instruction group supports register immediate operations. The immediate operand must be less than 6 bits in length. The instruction group has a 4 bit function field to support additional instruction encodings. 
	\begin{center}
		\begin{tabular}{|c|c|c|c|c|c|}
			\hline 
			& \multicolumn{5}{|c|}{\textbf{16 bit instruction encoding}} \\ 
			\hline
			\textbf{Mnemonic \& operands} & \textbf{OP code} & \textbf{Function 4} & \textbf{Operand register} & \multicolumn{2}{|c|}{\textbf{Immediate operand}} \\
			\hline 
			addi $ R_{1} $, $ imm $ & 110 & 1000 & xxx & \multicolumn{2}{|c|}{xxxxxx} \\ 
			\hline 
			subi $ R_{1} $, $ imm $ & 110 & 1001 & xxx & \multicolumn{2}{|c|}{xxxxxx} \\ 
			\hline 
			lui $ R_{1} $, $ imm $ & 011 & 0000 & xxx & \multicolumn{2}{|c|}{xxxxxx} \\ 
			\hline 
		\end{tabular} 
	\end{center}

	\newpage

	\subsection{B-Type instructions}
	The B-type instruction group contains branching instructions. The group encodes an 11 bit signed immediate used to specify branch distance relative to the PC. The instruction group has a 2 bit function field to support additional instruction encodings. 
	\begin{center}
		\begin{tabular}{|c|c|c|c|c|c|}
			\hline 
			& \multicolumn{5}{|c|}{\textbf{16 bit instruction encoding}} \\ 
			\hline
			\textbf{Mnemonic \& operands} & \textbf{OP code} & \textbf{Function 2} & \multicolumn{3}{|c|}{\textbf{Immediate address operand}} \\
			\hline 
			beq $ imm $ & 001 & 00 & \multicolumn{3}{|c|}{xxxxxxxxxxx} \\ 
			\hline 
			bne $ imm $ & 001 & 01 & \multicolumn{3}{|c|}{xxxxxxxxxxx} \\ 
			\hline 
			bgt $ imm $ & 001 & 10 & \multicolumn{3}{|c|}{xxxxxxxxxxx} \\ 
			\hline 
			blt $ imm $ & 001 & 11 & \multicolumn{3}{|c|}{xxxxxxxxxxx} \\ 
			\hline 
		\end{tabular}
	\end{center}

	\subsection{S-Type instructions}
	The S-type instruction group contains system commands. The group consist only of an opcode field followed by a 13 bit function field.
	\begin{center}
		\begin{tabular}{|c|c|c|c|c|c|}
			\hline 
			& \multicolumn{5}{|c|}{\textbf{16 bit instruction encoding}} \\ 
			\hline
			\textbf{Mnemonic \& operands} & \textbf{OP code} & \multicolumn{4}{|c|}{\textbf{Function 13}} \\
			\hline 
			hlt & 000 & \multicolumn{4}{|c|}{0000000000000} \\ 
			\hline 
		\end{tabular}
	\end{center}
\end{par}

\newpage

\section{Programmers assembly reference}
\label{progref}
The table below list all available assembly instructions. This table serves as a quick reference for system programmers. \nameref{id} section should be referenced for the assembly instructions binary encoding. 
\begin{par}
	\begin{center}
		\begin{tabular}{|c|c|c|}
			\hline
			\textbf{No.} & \textbf{Mnemonic} & \textbf{Description} \\
			\hline 
			1 & hlt & Halts program execution \\
			\hline 
			2 & add $ R_{1} $, $ R_{2} $, $ R_{3} $ & adds contents of registers $ R_{2} $ and $ R_{3} $ result stored in $ R_{1} $ \\
			\hline
			3 & sub $ R_{1} $, $ R_{2} $, $ R_{3} $ & subtracts contents of registers $ R_{2} $ and $ R_{3} $ result stored in $ R_{1} $ \\
			\hline
			4 & and $ R_{1} $, $ R_{2} $, $ R_{3} $ & logical and the contents of registers $ R_{2} $ and $ R_{3} $ result stored in $ R_{1} $ \\
			\hline
			5 & or $ R_{1} $, $ R_{2} $, $ R_{3} $ & logical or the contents of registers $ R_{2} $ and $ R_{3} $ result stored in $ R_{1} $ \\
			\hline
			6 & xor $ R_{1} $, $ R_{2} $, $ R_{3} $ & logical xor the contents of registers $ R_{2} $ and $ R_{3} $ result stored in $ R_{1} $ \\
			\hline
			7 & nand $ R_{1} $, $ R_{2} $, $ R_{3} $ & logical nand the contents of registers $ R_{2} $ and $ R_{3} $ result stored in $ R_{1} $ \\
			\hline
			8 & lw $ R_{1} $, $ R_{2} $ & 16-bit word in memory at address $ R_{2} $ loaded in register $ R_{1} $ \\
			\hline
			9 & sw $ R_{1} $, $ R_{2} $ & contents of register $ R_{1} $ stored at memory address $ R_{2} $ \\
			\hline
			10 & asr $ R_{1} $, $ R_{2} $ & arithmetic shift right $ R_{2} $ store result in $ R_{1} $ \\
			\hline
			11 & asl $ R_{1} $, $ R_{2} $ & arithmetic shift left $ R_{2} $ store result in $ R_{1} $ \\
			\hline
			12 & cmp $ R_{1} $, $ R_{2} $ & compare register $ R_{1} $ with register $ R_{2} $ results stored in status register \\
			\hline
			13 & jalr $ R_{1} $, $ R_{2} $ & jump to address in register $ R_{2} $ store current address + 1 in register $ R_{1} $ \\
			\hline
			14 & push $ R_{1} $ & push contents of register $ R_{1} $ onto stack \\
			\hline
			15 & pop $ R_{1} $ & pop 16-bit word off stack into register $ R_{1} $ \\
			\hline
			16 & lsp $ R_{1} $ & (load stack pointer) contents of register $ R_{1} $ transferred into stack pointer \\
			\hline
			17 & addi $ R_{1} $, $ imm $ & add immediate to $ R_{1} $ store result in $ R_{1} $  \\
			\hline
			18 & subi $ R_{1} $, $ imm $ & subtract immediate from $ R_{1} $ store result in $ R_{1} $  \\
			\hline
			19 & lui $ R_{1} $, $ imm $ & load upper immediate 10 bits then zero bottom 6 bits of register $ R_{1} $. \\
			\hline
			20 & beq $ imm $ & branch if last comparison instruction (No. 12) indicated $ R_{1} $ == $ R_{2} $ \\
			\hline
			21 & bne $ imm $ & branch if last comparison instruction (No. 12) indicated $ R_{1} $ != $ R_{2} $ \\
			\hline
			22 & bgt $ imm $ & branch if last comparison instruction (No. 12) indicated $ R_{1} > R_{2} $ \\
			\hline
			23 & blt $ imm $ & branch if last comparison instruction (No. 12) indicated $ R_{1} < R_{2} $ \\
			\hline
		\end{tabular}
	\end{center}
\end{par}

\newpage

\section{Common C language constructs compiled in assembly}
Listed below are common C language constructs and how they would be implemented in the Shell ISA. \\
\textbf{Note:} Information inside asterisks is a comment. 
\label{commonC}
\begin{par}
	
	\subsection{Assignment}
	\begin{center}
		\begin{tabular}{|l|l|}
			\hline
			x = 5; & xor r3, r3, r3 \\
			& addi r3, 5 \\
			\hline
		\end{tabular}
	\end{center}

	\subsection{Addition}
	\begin{center}
		\begin{tabular}{|l|l|}
			\hline
			x = x + y; & add r3, r3, r4 \\
			x = y + y; & add r3, r4, r4 \\
			y = x + x; & add r4, r3, r3 \\
			\hline
		\end{tabular}
	\end{center}

	\subsection{Subtraction}
	\begin{center}
		\begin{tabular}{|l|l|}
			\hline
			x = x - y; & sub r3, r3, r4 \\
			x = y - y; & sub r3, r4, r4 \\
			y = x - x; & sub r4, r3, r3 \\
			\hline
		\end{tabular}
	\end{center}
	
	\subsection{Logical operations}
	\begin{center}
		\begin{tabular}{|l|l|}
			\hline
			x = x \& y; & and r3, r3, r4 \\
			x = x $ \vert $ y; & or r3, r3, r4 \\
			x = y \^{} y; & xor r3, r3, r4 \\
			y = \~{} (y \& x); & nand r3, r3, r4 \\
			\hline
		\end{tabular}
	\end{center}

	\subsection{For loop control structure}
	\begin{center}
		\begin{tabular}{|l|l|}
			\hline
			int i; & xor r1, r1, r1 \\
			int x = 0; & xor r2, r2, r2 \\
			for (i = 0; i $ < $ 3; i++) \{ & xor r3, r3, r3 \\
			x = x + 1; & addi r3, 3 \\
			\} & addi r2, 1 \\
			& addi r1, 1  \\
			& cmp r1, r3 \\
			& blt -3 \\
			\hline
		\end{tabular}
	\end{center}

	\subsection{while loop control structure}
	\begin{center}
		\begin{tabular}{|l|l|}
			\hline
			int x = 0; & xor r1, r1, r1 \\
			while(1) \{ & xor r2, r2, r2 \\
			x = x + 1; & addi r2, 3 \\
			\} & addi r1, 1 \\
			& jalr r7, r2 \\
			\hline
		\end{tabular}
	\end{center}

	\subsection{if-else control structure}
	\begin{center}
		\begin{tabular}{|l|l|}
			\hline
			if (x == 0) \{ & cmp r1, r0 \\
				*do this* & bnq 5 \\
			\} else \{ & *do this* \\
				*do this* & xor r3, r3, r3 \\
			\} & addi r3, 8 \\
			& jalr r7, r3 \\
			& *do this* \\
			\hline
		\end{tabular}
	\end{center}

	\subsection{Relational operator "=="}
	\begin{center}
		\begin{tabular}{|l|l|}
			\hline
			if (x == y) \{ & cmp r1, r2 \\
			\} & beq *PC relative immediate address of a label* \\
			\hline
		\end{tabular}
	\end{center}

	\subsection{Relational operator "!="}
	\begin{center}
		\begin{tabular}{|l|l|}
			\hline
			if (x != y) \{ & cmp r1, r2 \\
			\} & bne *PC relative immediate address of a label* \\
			\hline
		\end{tabular}
	\end{center}

	\subsection{Relational operator less than}
	\begin{center}
		\begin{tabular}{|l|l|}
			\hline
			if (x $ < $ y) \{ & cmp r1, r2 \\
			\} & blt *PC relative immediate address of a label* \\
			\hline
		\end{tabular}
	\end{center}

	\subsection{Relational operator greater than}
	\begin{center}
		\begin{tabular}{|l|l|}
			\hline
			if (x $ > $ y) \{ & cmp r1, r2 \\
			\} & bgt *PC relative immediate address of a label* \\
			\hline
		\end{tabular}
	\end{center}

	\subsection{Relational operator greater than or equal to}
	\begin{center}
		\begin{tabular}{|l|l|}
			\hline
			if (x $ >= $ y) \{ & cmp r1, r2 \\
			\} & bgt *PC relative immediate address of a label* \\
			& bge *PC relative immediate address of a label* \\
			\hline
		\end{tabular}
	\end{center}

	\subsection{Relational operator less than or equal to}
	\begin{center}
		\begin{tabular}{|l|l|}
			\hline
			if (x $ <= $ y) \{ & cmp r1, r2 \\
			\} & blt *PC relative immediate address of a label* \\
			& bge *PC relative immediate address of a label* \\
			\hline
		\end{tabular}
	\end{center}

	\subsection{Functions (call and return) pass by value}
	\begin{center}
		\begin{tabular}{|l|l|}
			\hline
			int foo(int a, int b) \{ & xor r1, r1, r1 \\
				return a + b; & xor r2, r2, r2 \\
			\} & xor r3, r3, r3 \\
			int main() \{ & addi r1, 3 \\
				foo(3, 3); & addi r2, 3 \\
			\} & addi r3, 50 \\
			& lsp r3 \\
			& xor r3, r3, r3 \\
			& addi r3, *calculated address of foo label* \\
			& push r1 \\
			& push r2 \\
			& jalr r7, r3 \\
			& pop r1 \\
			& xor r3, r3, r3 \\
			& addi r3, *calculated address of exit label* \\
			& jalr r7, r3 \\
			& foo: \\
			& pop r1 \\
			& pop r2 \\
			& add r1, r1, r2 \\
			& push r1 \\
			& jalr r7, r7 \\
			& exit: \\
			\hline
		\end{tabular}
	\end{center}

	\subsection{Functions (call and return) pass by reference}
	\begin{center}
		\begin{tabular}{|l|l|}
			\hline
			void foo(int* a, int* b) \{ & xor r1, r1, r1 \\
				*a = *a + *b; & xor r2, r2, r2 \\
			\} & addi r1, 3 \\
			& addi r2, 3 \\
			int main() \{ & xor r3, r3, r3 \\
				int a = 3; & addi r3, 50 \\
				int b = 3; & lsp r3 \\
				foo(\&a, \&b); & xor r3, r3, r3 \\
			\} & sw r1, r3 \\
			& xor r4, r4, r4 \\
			& addi r4, 1 \\
			& sw r2, r4 \\
			& xor r2, r2, r2 \\
			& addi r2, *calculated address of foo label* \\
			& push r3 \\
			& push r4 \\
			& jalr r7, r2 \\
			& xor r3, r3, r3 \\
			& addi r3, *calculated address of exit label* \\
			& jalr r7, r3 \\
			& foo: \\
			& pop r1 \\
			& pop r2 \\
			& lw r3, r1 \\
			& lw r4, r2 \\
			& add r3, r3, r4 \\
			& sw r3, r1 \\
			& jalr r7, r7 \\
			& exit: \\
			\hline
		\end{tabular}
	\end{center}

\end{par}

\section{Instruction usage and justification.}
\label{insuj}
\begin{par}
	\begin{center}
		\begin{tabular}{|c|c|c|}
			\hline
			\textbf{No.} & \textbf{Mnemonic} & \textbf{Usage and justification} \\
			\hline 
			1 & hlt & Halts program execution \\
			\hline 
			2 & add $ R_{1} $, $ R_{2} $, $ R_{3} $ & add two 16 bit words \\
			\hline
			3 & sub $ R_{1} $, $ R_{2} $, $ R_{3} $ & subtract two 16 bit words \\
			\hline
			4 & and $ R_{1} $, $ R_{2} $, $ R_{3} $ & perform logical and between registers \\
			\hline
			5 & or $ R_{1} $, $ R_{2} $, $ R_{3} $ & perform logical or between registers \\
			\hline
			6 & xor $ R_{1} $, $ R_{2} $, $ R_{3} $ & perform logical xor between registers \\
			\hline
			7 & nand $ R_{1} $, $ R_{2} $, $ R_{3} $ & perform logical nand between registers \\
			\hline
			8 & lw $ R_{1} $, $ R_{2} $ & load 16-bit word form memory \\
			\hline
			9 & sw $ R_{1} $, $ R_{2} $ & store 16-bit word in memory \\
			\hline
			10 & asr $ R_{1} $, $ R_{2} $ & perform an arithmetic shift right on register (reg / 2) \\
			\hline
			11 & asl $ R_{1} $, $ R_{2} $ & perform an arithmetic shift left on register (reg * 2) \\
			\hline
			12 & cmp $ R_{1} $, $ R_{2} $ & compare two 16-bit words \\
			\hline
			13 & jalr $ R_{1} $, $ R_{2} $ & unconditional jump to a subroutine store return address in register \\
			\hline
			14 & push $ R_{1} $ & push 16-bit word onto stack \\
			\hline
			15 & pop $ R_{1} $ & pop 16-bit word off stack \\
			\hline
			16 & lsp $ R_{1} $ & initialize the stack pointer \\
			\hline
			17 & addi $ R_{1} $, $ imm $ & add an immediate to a register  \\
			\hline
			18 & subi $ R_{1} $, $ imm $ & subtract an immediate from a register  \\
			\hline
			19 & lui $ R_{1} $, $ imm $ & load upper 10 bits of a register and zero bottom six bits \\
			\hline
			20 & beq $ imm $ & PC relative branch if two compared registers were equal \\
			\hline
			21 & bne $ imm $ & PC relative branch if two compared registers were not equal \\
			\hline
			22 & bgt $ imm $ & PC relative branch if a register is greater than another register \\
			\hline
			23 & blt $ imm $ & PC relative branch if a register is less than another register \\
			\hline
		\end{tabular}
	\end{center}
\end{par}

\end{document}